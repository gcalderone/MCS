% ----------------------------------------------------------------------^
% Copyright (C) 2004, 2005, 2006, 2007, 2008 Giorgio Calderone
% (mailto: <gcalderone@ifc.inaf.it>)
% 
% This file is part of MCS.
% 
% MCS is free software; you can redistribute it and/or modify
% it under the terms of the GNU General Public License as published by
% the Free Software Foundation; either version 2 of the License, or
% (at your option) any later version.
% 
% MCS is distributed in the hope that it will be useful,
% but WITHOUT ANY WARRANTY; without even the implied warranty of
% MERCHANTABILITY or FITNESS FOR A PARTICULAR PURPOSE.  See the
% GNU General Public License for more details.
% 
% You should have received a copy of the GNU General Public License
% along with MCS; if not, write to the Free Software
% Foundation, Inc., 51 Franklin St, Fifth Floor, Boston, MA  02110-1301  USA
% 
% ----------------------------------------------------------------------$
\chapter{Administrator's manual}
In questo capitolo descriviamo le procedure per la configurazione,
installazione e manutenzione di un'applicazione basata su
\mcs. In particolare ci occuperemo di:
\begin{itemize}
\item Creare l'ambiente di esecuzione dell'applicazione;
\item Scrivere il codice necessario per adattare \mcs ad una
  applicazione specifica;
\item Eseguire e mantenere tale applicazione;
\end{itemize}


\section{Dipendenze}\label{sec:dipendenze}
Per creare una nuova applicazione basata su \mcs bisogna
innanzitutto installare le librerie ed i programmi necessari, di cui
riportiamo l'elenco:
\begin{itemize}
\item
  \verb|CommonC++|\footnote{http://sourceforge.net/projects/cplusplus};
\item \verb|pcre|
  \footnote{ftp://ftp.csx.cam.ac.uk/pub/software/programming/pcre};
\item \verb|mysqlclient|\footnote{www.mysql.com};
\item \verb|perl|: \footnote{www.perl.com};
\end{itemize}
%
Quindi bisogna assicurarsi che ci sia un server MySQL in esecuzione e
che l'eseguibile di perl sia nel path per l'esecuzione.

\section{Installazione di MCS}
Il codice di \mcs viene fornito sotto forma di file
\verb|tar.gz|. Esso va scaricato\footnote{il sito di riferimento \`e:
...} e scompattato in una directory di propriet\`a dell'amministratore
dell'applicazione. Prima di effettuare la compilazione \`e opportuno
consultare la Sez.\ref{sec:dipendenze} per controllare che tutte le
dipendenze siano soddisfatte. \`E possibile quindi eseguire i consueti
comandi:
\begin{itemize}
\item \verb|configure|;
\item \verb|make|;
\item \verb|make install| (richiede i permessi di amministratore);
\end{itemize}
%
Con i primi due si compila il codice sorgente della libreria
\mcs. Con il secondo si installano i file necessari all'utilizzo
della libreria stessa. In particolare viene installato nella directory
\verb|/bin| il programma \verb|mcs-config| che pu\`o essere usato per
ricavare i percorsi in cui sono memorizzati i file da includere e la
libreria, nonch\`e le opzioni utilizzate per la compilazione della
libreria \mcs. \`E consigliabile utilizzare le stesse opzioni
per la compilazione del codice utente. Il command :

\bigskip

\verb|mcs-config --help|
\bigskip\newline
%
fornisce una guida per l'utilizzo di \verb|mcs-config|. Vengono
inoltre installati (per default in \verb|/usr/local/share/mcs|, pi\`u
avanti ci riferiremo a tale directory con \verb|sharedir|) alcuni
script, i programmi client in perl e IDL, i programmi di esempio e una
struttura di directory con l'ambiente per le nuove applicazioni.

\subsection{Switch di precompilazione}\label{ssec:switch}
\`E possibile modificare alcuni switch di precompilazione nel codice
che attivano o meno alcune funzionalit\`a. Tali switch si trovano nel
file \verb|src/mcs_switch.h| della directory di distribuzione, essi
vanno commentati o meno a seconda che si voglia o meno quella
funzionalit\`a nel application server \mcs. Segue un elenco con
descrizione dei singoli switch:
\begin{enumerate}
\item \verb|MCS_HAVE_DBH_CMD|: attiva il supporto per il command
  \verb|DBH|, per il collegamento ad un server MySQL che non viene
  eseguito sullo stesso computer del server \mcs, tale switch va
  attivato solo se strettamente necessario;
\item \verb|MCS_HAVE_SYS_CMD|: attiva il supporto per il
  command \verb|SYS|, per l'esecuzione remota di comandi sul computer
  che esegue il server \mcs, tale switch va attivato solo se
  strettamente necessario;
\end{enumerate}


\section{Quick-start per la creazione di una nuova applicazione}
Prima di procedere alla descrizione dettagliata del codice sorgente,
delle tabelle e delle directory che costituiscono l'ambiente di
esecuzione, illustriamo una procedura di quick-start che permette di
configurare automaticamente lo stesso per una nuova
applicazione. Tenendo presente che lo sviluppo di una nuova
applicazione basata su \mcs richiede una lettura attenta di
questo manuale, facciamo notare che nell'utilizzo comune il primo
passo da eseguire \`e proprio quello di utilizzare tale procedura, per
sollevare l'amministratore da compiti molto ripetitivi. Tale procedura
consiste nell'esecuzione del programma \verb|mcs_newapp|. Si tratta di
uno script interattivo che raccoglie informazioni sull'applicazione
che si vuole creare, e configura l'ambiente per le'esecuzione. In
particolare tale script si occupa di:

\begin{itemize}
\item copiare i file sorgente minimali da utilizzare come base di
  partenza;
\item creare le directory necessarie (opzionale);
\item creare le tabelle necessarie nel database (opzionale);
\item creare l'account utente dell'amministratore dell'applicazione,
  come utente del database (opzionale, in tal caso viene richiesta la
  password dell'account root del database);
\end{itemize}
%
Infine vengono fornite alcune informazioni circa i file sorgente da
modificare per l'adattamento di \mcs ed o comandi da utilizzare
per la compilazione e l'utilizzo dell'applicazione.


\section{Descrizione dell'ambiente}\label{prep_ambiente}
Richiamando il paragone tra un servizio offerto da una applicazione
basata su \mcs ed una shell, ci rendiamo conto che tale
applicazione necessita di un ambiente di esecuzione. La creazione di
tale ambiente \`e facoltativa, ma si rende necessaria se si vogliono
sfruttare alcune caratteristiche di \mcs fra le quali:

\begin{itemize}
\item esecuzione di script e programmi;
\item scrittura di log file;
\item disponibilit\`a di una work directory per l'utente;
\item possibilit\`a di gestire i permessi utente;
\item possibilit\`a di memorizzare variabili di ambiente su database;
\end{itemize}
%
L'ambiente di esecuzione consiste in una struttura di directory ed una
serie di tabelle di database che verranno descritte di seguito, che
possono essere create automaticamente tramite lo script
\verb|mcs_newapp| descritto precedentemente. Al fine di descrivere
l'ambiente di esecuzione di un'applicazione basata su \mcs
elenchiamo i passi necessari alla preparazione dello stesso:

\begin{enumerate}
\item Creare un account utente di sistema e la sua home
  directory. Tale utente rappresenta l'amministratore del sistema
  informativo e tutte le informazioni relative verranno salvate nella
  home directory, quindi \`e necessario che l'accesso ad essa sia
  precluso a tutti gli altri utenti del computer. Pi\`u avanti ci
  riferiremo a questo utente come \verb|SYS_USER|;
\item Creare un account utente MySQL, non \`e necessario che abbia lo
  stesso nome dell'account di sistema. Tale utente dovr\`a avere
  accesso soltanto al database di cui al punto successivo. \`E
  consigliabile utilizzare due password diverse per l'account di
  sistema e l'account MySQL. Pi\`u avanti ci riferiremo a questo
  utente come \verb|DB_USER|;
\item Crere un database MySQL che manterr\`a tabelle di configurazione
  e dati. Tale database dovrebbe avere un nome che ricorda
  l'applicazione. Pi\`u avanti ci riferiremo a questo utente come
  \verb|DB_NAME|. L'accesso a questo database dovr\`a essere
  consentito soltanto all'utente MySQL di cui al punto precedente ed
  agli utenti che accederanno al servizio. \`E altres\`i necessario
  che il permesso di accedere per utenti MySQL siano consentiti
  soltanto da '\verb|localhost|, questo perch\`e in realt\`a ad
  accedere sar\'a il software \mcs stesso che supponiamo sia in
  esecuzione sullo stesso computer;
\item Creare una directory, sotto la home directory per la
  memorizzazione dei file relativi all'applicazione. Tale directory
  dovrebbe avere un nome che ricorda l'applicazione. Ribadiamo che
  soltanto l'account \verb|SYS_USER| deve avere accesso a queste
  directory. Pi\`u avanti ci riferiremo a questa directory come
  \verb|APPD|, Application Directory;
\item Creare una directory (usualmente \verb|APPD/src|) che conterr\`a
  i file sorgente della nuova applicazione. \`E possibile copiare i
  file presenti in \verb|sharedir/examples/mcstest| come punto di
  partenza. Pi\`u avanti ci riferiremo a questa directory come
  \verb|SRCD|, Source Directory;
\end{enumerate}
%
Questo completa la prima parte della configurazione dell'ambiente. Di
seguito bisogner\`a creare le tabelle del database e le directory che
conterranno i programmi utilizzati dall'applicazione, nonch\`e le work
directory degli utenti. Tali passi sono descritti nelle prossime
sezioni.

\subsection{Preparazione delle tabelle dei permessi e di ambiente}
Le tabelle necessarie al funzionamento di \mcs sono:
\verb|settings| e \verb|grants|, di cui riportiamo le istruzioni SQL
per la creazione:
\begin{verbatim}
create table settings (
  id      integer unsigned not null auto_increment,
  fld     char(10),
  val     char(15),
  type    char(1) ,
  primary key (id),
  unique key (fld)
);

create table grants (
  id    integer unsigned not null auto_increment,
  username char(20),
  grants int unsigned default 0,
  primary key (id)
);
\end{verbatim}
%
Nella tabella \verb|settings| \`e possibile memorizzare variabili che
regolano il funzionamento del sistema, alla stessa maniera delle
variabili di ambiente per le shell. Tale tabella verr\`a caricata in
memoria dall'oggetto \verb|MCSEnv| ed i valori possono essere
recuperati utilizzando il metodo \verb|MCSEnv::var|. \`E possibile
utilizzare le seguenti variabili che verranno lette all'avvio del
programma:

\begin{itemize}
\item \verb|PORT|: porta su cui il server \mcs si metter\`a in
  ascolto;
\end{itemize}

Un esempio delle istruzioni per creare tali variabili \`e:
\begin{verbatim}
insert into settings values(NULL, 'PORT', '6523', 'N');
\end{verbatim}
%
Nella tabella \verb|grants| saranno memorizzati i diritti dei singoli
utenti, specificando per ogni record il nome utente e la somma dei
seguenti simboli, a seconda di quali diritti si vogliono assegnare. I
singoli diritti attualmente riguardano la possibilit\`a di eseguire un
certo command, la lista completa \`e la seguente:

\begin{verbatim}
#define MG_NO_GRANTS       0 - nessun diritto
#define MG_LOGIN           1 - login consentito
#define MG_SQL_SCRIPTS     2 - command SQL
#define MG_SCRIPTS         4 - command SCR
#define MG_QUERY           8 - command QRY
#define MG_BATCH          16 - command BAT
#define MG_GET            32 - command GET
#define MG_PUT            64 - command PUT
#define MG_SYS           128 - command SYS
#define MG_ADMIN         256 - command ADM
#define MG_ALL           511 - tutti i diritti
\end{verbatim}
%
Questi simboli sono definiti nel file \verb|SRCD/src/settings.h|.

\subsection{Preparazione delle directory dell'applicazione}
Riferendoci al paragone tra un'applicazione basata su \mcs e una
shell di sistema, facciamo notare che anche in questo caso abbiamo
bisogno di directory di lavoro, per i programmi utente, per i
programmi dell'amministratore, ecc... L'elenco seguente riporta i nomi
che devono essere assegnati a tali directory che vanno create sotto
\verb|APPD|:
\begin{itemize}
\item \verb|adm/scr|: eseguibili per l'amministratore;
\item \verb|adm/sql|: script SQL per l'amministratore;
\item \verb|input|: file per il caricamento dei dati nel database;
\item \verb|log|: contiene i file di log
\item \verb|usr/scr|: eseguibili per gli utenti;
\item \verb|usr/sql|: script SQL per gli utenti;
\item \verb|work|: work directory degli utenti
\end{itemize}
%
Esiste gi\`a una struttura di directory pronta nella directory
\verb|sharedir/env| che pu\`o essere copiata direttamente in
\verb|APPD|.

\subsection{Aggiunta di utenti}
Per permettere agli utenti di accedere al sistema informativo bisogna
inserirli come utenti MySQL, per cui si rimanda alla documentazione
relativa per questo aspetto. \`E importante che agli utenti registrati
sia data l'opportunit\`a di accedere soltanto dall'indirizzo IP
dell'host su cui viene eseguito l'application server, soltanto al
database legato all'applicazione, e di non poter modificare le tabelle
\verb|grants| e \verb|settings|. Quindi bisogna inserire un record
nella tabella \verb|grants|, come descritto di seguito, nell'ipotesi
che l'utente si chiami \verb|newuser| e che si vogliano dare tutt i
permessi tranne quelli di \verb|SYS| e \verb|ADMIN|:

\begin{verbatim}
insert into grants values (NULL, 'user', 127);
\end{verbatim}


\section{Descrizione del codice sorgente}
L'application server \mcs \`e un server TCP scritto in C++ con
differenti caratteristiche quali le capacit\`a di: comunicare con un
server di database, interagire con programmi esterni tramite riga di
command e standard output, rispondere a comandi mnemonici inviati
dall'utente con un protocollo di comunicazione semplice ma non
limitato, ecc... Per creare una nuova applicazione bisogner\`a
comunque scrivere del codice specifico, prima di dedicarci
all'adattamento di \mcs \`e per\`o conveniente familiarizzare
con il codice sorgente di \mcs stesso, in particolare sulle
classi ed i metodi utilizzati. Di seguito verr\`a fornita una breve
descrizione della gerarchia di classi, ma per uno studio pi\`u attento
rimandiamo alla documentazione tecnica presente in
\verb|sharedir/doc/html|.

\subsection{File sorgente}
Segue una breve descrizione dei singoli file sorgente presenti nella
distribuzione ed il loro utilizzo:
\begin{itemize}
\item \verb|mcs.hh|: contiene le definizioni delle classi e costanti
  necessarie al funzioanmento del server. Questo \`e il file che va
  incluso nei programmi utente;
\item \verb|MCS*.cc|: questi file contengono il codice sorgente che
  imeplementa le classi;
\item \verb|mcs_globals.hh|: contiene le dichiarazione delle variabili
  globali;
\item \verb|mcs_switch.h|: contiene gli switch che abilitano o meno
  alcune funzionalit\`a nel server;
\item \verb|mcs_user.hh|: contiene l'implementazione delle funzioni
  \verb|MCS_NewClientThread| e \verb|MCS_NewRootThread|, richiamate
  dalla classe \verb|MCSServer| per creare i thread client e root
  rispettivamente.
\end{itemize}


\subsection{I Thread e le classi}
Questo software fa largo uso dei thread. I thread sono essenzialmente
diverse copie dello stesso software che girano come un unico
processo. Questo significa che una volta lanciato il programma
verranno generati altri sotto programmi che lavoreranno in
parallelo. Un esempio semplice riguarda i thread per servire gli
utenti: ogni volta che un nuovo utente si collega verr\`a generato un
nuovo thread e ad esso verr\`a affidata la gestione della
comunicazione con il nuovo utente, mentre i restanti thread possono
occuparsi degli altri utenti o di altro ancora. Facciamo notare che
senza il multithreading un programma del genere sarebbe stato molto
pi\`u difficile da implementare.
%
Per sfruttare il multithreading viene utilizzata la classe
\verb|Thread| della libreria \verb|CommonC++|, essa si occupa di
creare un nuovo thread quando richiesto e gestisce la chiamata dei
suoi metodi. Il metodo pi\`u importante della classe \verb|Thread| \`e
\verb|run()| ovvero il corpo dell'esecuzione del thread. Non appena il
thread viene creato esso comincia automaticamente ad eseguire il
metodo \verb|run()|, e si autodistrugge appena tale metodo
ritorna. Esiste ovviamente anche la possibilit\`a di terminare il
thread da parte dell'amministratore del sistema.
%
Esistono quattro tipi diversi di classi che derivano da \verb|Thread|,
alcune di esse vengono utilizzate come classi base per
l'implementazione di classi pi\`u specializzate, altre vengano
utilizzate per istanziare oggetti che girano in thread separati:
\begin{enumerate}
\item \verb|MCSThread|: questa classe non viene mai utilizzata come
  thread, ma soltanto come base per generare classi pi\`u
  specifiche. Essa implementa il meccanismo di log e tracciamento
  degli errori utilizzato nelle classi derivate. Tutte le classi
  descritte di seguito derivano infatti da questa che a sua volta
  deriva dalla classe \verb|Thread|. Nota che questa classe non
  implementa il metodo virtuale \verb|run()|, che deve quindi essere
  implementato in una classe derivata;
\item \verb|MCSCLient|: questa \`e la classe base che si occupa di
  gestire la comunicazione con gli utenti. In questo caso abbiamo
  implementato il metodo \verb|run()| come un ciclo che legge dal
  socket i comandi inviati dall'utente, li elabora e quindi invia una
  risposta. Il ciclo termina quando il client invia un command di
  \verb|bye|. Per creare una nuova applicazione bisogna derivare
  questa classe e, sfruttando gli hook (vedi Sez. \ref{ssec:hook}) o i
  metodi virtuali personalizzare il suo funzionamento.
\item \verb|MCSRoot|: si tratta di una classe che deriva da
  \verb|MCSCLient|, viene utilizzato per le operazioni locali che sono
  indipendenti dalle richieste degli utenti. Un tipico esempio di
  utilizzo di questa classe \`e il caricamento di dati nel database e
  la manutenzione ed elaborazione degli stessi. Anche in questo caso
  per la personalizzazione dell'applicazione bisogner\`a derivare tale
  classe sfruttando i suoi hook o i metodi virtuali.
\item \verb|MCSServer|: questa classe si occupa di gestire il socket
  che accetta le richieste di connessione verso l'esterno. Quando un
  nuovo utente tenta di collegarsi al sistema esso genera un nuovo
  client thread ed assegna ad esso la gestione della nuova
  connessione. Alla creazione dell'oggetto \verb|MCSServer| viene
  automaticamente creato un root thread. Quando il thread
  \verb|MCSServer| termina si occupa prima di distruggere tutti i
  thread precedentemente generati. Per generare i client thread e root
  thread questa classe utilizza le routine \verb|MCS_NewClientThread|
  e \verb|MCS_NewRootThread|. Tali routine sfruttano due simboli
  (\verb|MCS_USERCLASS| e \verb|MCS_ROOTCLASS|) che
  devono essere definiti dall'utente ed indicano quali classi usare
  rispettivamente per i thread client e root.
\end{enumerate}
%
Per meglio comprendere le sezioni successive dedicate agli hook e gli
esempi riportiamo di seguito la descrizione di alcuni metodi della
classe \verb|MCSClient|:
\begin{enumerate}
\item \verb|Send|: utilizzato per inviare un messaggio all'utente e
  contemporaneamente effettuare il log dello stesso. Nota che il primo
  parametro deve seguire la convenzione dei valori di ritorno
  (descritta nella Sez. \ref{ssvalrit}), mentre gli altri seguono la
  convenzione di chiamate tipo \verb|printf|, ovvero una stringa di
  formato e le eventuali variabili;
\item \verb|Log|: utilizzato per effettuare il log di un
  messaggio. Nota che il primo parametro deve seguire la convenzione
  dei valori di ritorno (descritta pi\`u avanti), mentre gli altri
  seguono la convenzione di chiamate tipo \verb|printf|, ovvero una
  stringa di formato e le eventuali variabili.
\item \verb|run|: questo \`e il metodo che viene richiamato come corpo
  del thread, \`e strutturato come un ciclo che legge dal socket di
  comunicazione una riga di command, lo esegue tramite la chiamata al
  metodo \verb|exec|, quindi se il command non \`e \verb|bye| invia un
  prompt all'utente ed attende il nuovo command;
\item \verb|exec|: questo metodo \`e il vero interprete dei comandi
  dell'utente. Esso accetta come parametro una stringa contenente il
  command, e tenta di interpretarlo nelle sue tre componenti (switch,
  command, parametri, vedi pi\`u avanti). Una volta identificato il
  command lo esegue con i parametri forniti e restituisce una risposta
  al client, quindi il metodo ritorna. Prima di iniziare
  l'interpretazione dei comandi richiama il metodo \verb|custom_exec|
  passando la stringa di command, cosicch\`e si possa modificare il
  comportamento del server creando nuovi comandi;
\item \verb|send_file|: con questo metodo \`e possibile inviare un
  file all'utente specificando il suo percorso;
\item \verb|send_queryres|: con questo metodo \`e possibile inviare
  all'utente il risultato di una query su database;
\end{enumerate}
%
Per ulteriori informazioni riferirsi alla documentazione tecnica.


\subsection{Altre classi}
Qui descriviamo le altre classi utilizzate nel software che non hanno
la funzionalit\`a dei thread.
\begin{enumerate}
\item \verb|MCSEnv|: questa classe viene utilizzata per leggere le
  informazioni di ambiente dal database, e fornirle ai thread che le
  richiedono, sollevando questi ultimi dal dover gestire questo
  accesso al database;
\item \verb|MCSHandleCmd|: si occupa di gestire le stringhe di
  command, in particolare viene usata per interpretare i comandi ed i
  parametri forniti;
\item \verb|MCSDB|: usata per collegarsi al database MySQL, effettuare
  query e leggere i risultati. Modificando questa classe \`e possibile
  accedere ad altri tipi di database e mantenere le stesse
  funzionalit\`a dell'application server, purch\`e si mantenga la
  stessa interfaccia;
\end{enumerate}

\subsection{Convenzione sui valori di ritorno}\label{ssvalrit}
I valori di ritorno dei metodi \verb|exec|, \verb|send_file|, ecc... e
deglki hook seguono una stessa convenzione che \`e quella di ritornare
un valore di tipo \verb|MCSRet|. Si tratta di una enumerazione con i
seguenti simboli:
\begin{enumerate}
\item \verb|OK  |: la routine  \`e stato eseguita con successo;
\item \verb|IERR|: la routine ha generato un errore ignorabile;
\item \verb|EERR|: la routine ha generato un errore grave;
\item \verb|KERR|: la routine ha generato un errore grave, ed il
  thread deve essere distrutto;
\end{enumerate}
%
Poich\`e l'esecuzione di qualunque command richiede che venga
successivamente chiamato il metodo \verb|Send| o \verb|Log| (si
ricorda che \verb|Send| chiama \verb|Log| al suo interno), tale valore
di ritorno deve anche essere usato come primo parametro per questa
chiamata.


\section{Adattamento di MCS}
L'adattamento di \mcs consiste nella personalizzazione del
funzionamento per l'applicazione che si vuole creare. Per far questo
bisogna derivare le classi che si occupano dei thread utenti e del
root thread, ovvero: \verb|MCSClient| e \verb|MCSRoot|, ed
implementare uno o pi\`u metodi virtuali o ``hooks'' (descritti nella
Sez. \ref{ssec:hook}). Bisogna ora definire i simboli
\verb|MCS_USERCLASS| e \verb|MCS_ROOTCLASS| con i nomi delle classi
che gestiranno rispettivamente i client e root threads, ed includere
il file \verb|mcs_user.hh|. Quindi bisogna istanziare un oggetto di
tipo \verb|MCSEnv| e specificare tutte le impostazioni necessarie
all'esecuzione del server. Infine bisogna istanziare un oggetto di
tipo \verb|MCSServer| passando al costruttore il riferimento
all'oggetto di tipo \verb|MCSEnv|.

\subsection{Gli ``hook''}\label{ssec:hook}
Derivando una classe \`e possibile aggiungere funzionalit\`a sotto
forma di metodi, oppure re-implementare quelli gi\`a esistenti. Nel
caso della classe \verb|MCSClient|, che si occupa di gestire le
sessioni utente, noi vogliamo aggiungere nuovi metodi ma vogliamo
anche interagire con quelli esistenti, poich\`e tutta le gestione di
una sessione si riduce ad un ciclo che legge da un socket (nel metodo
\verb|run|) e ad una chiamata al metodo \verb|exec| per eseguire il
command. Per questo motivo sono stati previsti specifici metodi
virtuali ``hook'' che nella implementazione di \mcs sono
semplicemente vuoti. Ma la chiamata a questi metodi si trova
all'interno del codice \mcs, quindi nel caso in cui questi
vengano implementati in una classe derivata essi verranno richiamati
al momento giusto. La classe \verb|MCSClient| presenta i seguenti
hook:
\begin{itemize}
\item \verb|MCSRet hk_connect()|: richiamato all'inizio del metodo
  \verb|run| ma prima che inizi il ciclo di lettura dal
  socket. Tramite questo metodo l'utente pu\`o istanziare eventuali
  oggetti necessari, oppure rifiutare la connessione ritornando un
  valore di errore. In tal caso \`e opportuno scrivere un messaggio di
  errore tramite il metodo \verb|Send| oppure \verb|Log|. Se la
  connessione \`e accettata bisogna ritornare il valore \verb|OK|;

\item \verb|MCSRet hk_exec(string& cmd, bool& cmd_executed)|: questo
  metodo viene richiamato all'interno del metodo \verb|exec| subito
  dopo l'analisi degli switch ma prima dell'analisi dei
  comandi. Se si vogliono aggiungere nuovi comandi a quelli di base
  bisogna implementare questo metodo. Il flag \verb|cmd_executed|
  viene utilizzato dal metodo \verb|exec| per sapere se il command \`e
  stato eseguito all'interno di \verb|hk_exec| o meno;

\item \verb|void hk_disconnect()|: questo metodo viene richiamato
  quando il metodo \verb|run| sta per terminare, conseguentemente al
  command di chiusura sessione da parte dell'utente. Per liberare
  memoria allocata \`e meglio utilizzare il distruttore della classe
  derivata.
\end{itemize}
%
Per la classe \verb|MCSRoot| al momento non \`e previsto alcun hook
poch\`e solitamente si deve implementare il metodo \verb|run| nella
classe derivata.

\subsection{Scrittura del codice}
In questa sezione riportiamo un esempio minimale di codice necessario
affinch\`e sia possibile creare una applicazione:

\begin{verbatim}
#include <mcs.hh>

//Derive a new client class
class myClient : public MCSClient
{
public:
  //empty constructor and destructor
  myClient(int lID, TCPSocket* server) : MCSClient(lID, server) {}
  ~myClient() {}

  //Hooking every received commands
  MCSRet hk_exec(string& cmd, bool& cmd_executed)
  {
    ...
    return OK;
  }
};

//Mandatory specification of which class we want to use as client and
//root thread
#define MCS_USERCLASS myClient
#define MCS_ROOTCLASS MCSRoot
#include <mcs_user.hh>


int main(int argc, char *argv[]) {
  //Create an environment with all switches set to false, then set
  //some properties
  MCSEnv* env=new MCSEnv(MCSEnv_SW_FALSE);
  env->appname="echosrv"; //--MANDATORY--
  env->port=9876;         //server will listen on port 9876
  env->cl_logstdout=true; //we want the log on stdout

  //Create a new server, note that we passed the environment reference
  //as paramter
  MCSServer* mcssrv=new MCSServer(env);
  mcssrv->start();   //...and start it

  //Wait for the server to terminate
  while (mcssrv->isRunning())
    Thread::sleep(500);

  delete env;
}
\end{verbatim}
%
Come si vede da questo codice minimale bisogna includere alcuni file
di \mcs, quindi, dopo aver derivato la classe \verb|MCSClient| e
definito i simboli \verb|MCS_USERCLASS| e \verb|MCS_ROOTCLASS|,
istanziare un oggetto di tipo \verb|MCSEnv|, settare almeno i campi
necessari (per una descrizione di questi ed altri campi vedi
Sez. \ref{sec:MCSEnv_fields}), ed infine istanziare l'oggetto di tipo
\verb|MCSServer| passando il riferimento all'oggetto
\verb|MCSEnv|. Quindi il programma attende finch\`e il server non
termina ed infine libera la memoria allocata. Il programma main
avrebbe anche potuto terminare il server con l'istruzione:

\bigskip
\verb|delete mcssrv;|
\bigskip\newline
%
Per altri esempi pi\`u completi riferirsi alla sezione degli esempi.


\subsection{Compilazione}
Durante l'installzione di \mcs viene copiato lo script
\verb|mcs-config| che fornisce le opzioni necessarie per compilare le
applicazioni utente. Se il programma utente si chiama \verb|prog.cc|
allora i comandi necessari per compilare e linkare le librerie sono i
seguenti:

\bigskip
\verb|g++ `mcs-config --cflags` -c prog.cc|

\verb|g++ -o prog prog.o `mcs-config --libs`|
\bigskip


\section{La classe MCSEnv}\label{sec:MCSEnv_fields}
Tale classe mantiene diverse informazioni relative al funzionamento
del server \mcs. Essa va istanziata, nella sua forma base o
derivata, prima di \verb|MCSServer|, quindi bisogna valorizzare i
campi necessari descritti pi\`u avanti, ed infine utilizzarla come
parametro per la costruzione di \verb|MCSServer|. I campi obbligatori
sono:
\begin{itemize}
\item \verb|glpath|: percorso completo di \verb|APPD|;
\item \verb|appname|: nome dell'applicazione;
\item \verb|appvers|: versione dell'applicazione;
\end{itemize}
%
Se inoltre il flags \verb|cl_have_db| \`e impostato a true bisogna
anche impostare i seguenti:
\begin{itemize}
\item \verb|db_user|: nome utente con il quale il root thread tenta
  l'accesso al DB;
\item \verb|db_pass|: password con la quale il root thread tenta
  l'accesso al DB;
\item \verb|db_name|: nome del database a cui accede il root thread;
\end{itemize}
%
Il costruttore della classe \verb|MCSEnv| necessita come parametro di
uno dei simboli \verb|MCSEnv_SW_FALSE| o \verb|MCSEnv_SW_DEFAULT| con
i quali tutti i restanti campi booleani ricevono un valore,
rispettivamente, falso o di default. Qui di seguito riportiamo
l'elenco ed il significato di tali campi, fra parentesi il loro valore
di default:

\begin{itemize}
\item \verb|cl_have_db|: indica se le funzionalit\`a del database
  debbano essere attivate o meno (true);
\item \verb|cl_verbose| (false): rende il server molto pi\`u prolisso per
  quanto riguarda il log. Questo switch \`e utile durante lo sviluppo
  per evidenziare eventuali problemi;
\item \verb|cl_netquiet| (false): se impostato a true vengono abolite
  tutte le risposte descrittive inviate ai client, verranno inviati
  soltanto i codici di risposta, i campi numerici e gli errori. Se il
  software del client rispetta fedelmente il protocollo di
  comunicazione sar\`a in grado di funzionare ugualmente in questo
  caso;
\item \verb|cl_logfile| (false): specifica se utilizzare o meno il log
  file;
\item \verb|cl_logstdout| (false): specifica se scrivere il log anche
  su stdout;
\item \verb|cl_rootlogin| (false): specifica se il root thread
  tenter\`a l'accesso al database;
\item \verb|cl_clean_logout| (false): specifica se quando l'utente
  effettua il logout la sua work directory va cancellata;
\item \verb|cl_autoexec| (false): specifica se quando un utente accede
  verr\`a automaticamente eseguito il command \verb|bat auto|;
\item \verb|cl_work| (true): specifica se verr\`a creata la work
  directory per l'utente quando questo si collega;
\item \verb|cl_local_scripts| (true): specifica se verr\`anno definiti i
  comandi \verb|SCR| e \verb|BAT| per l'utilizzo di script;
\end{itemize}



\section{Descrizione formato del log}
Il log generato dall'application server fornisce una serie di
informazioni utili ad effettuare controlli di debug, di sicurezza,
statistiche ecc... sulle operazioni effettuate dagli utenti. Il log
pu\`o avvenire su stdout e/o su file, a seconda delle
impostazioni. Ogni riga rappresenta una azione compiuta
dall'application server. I vari campi indicano l'anno, il mese, il
giorno, l'ora, i minuti ed i secondi in cui avviene
l'operazione. Quindi troviamo un numero progressivo univoco (modulo
1000000) utile per avere l'esatta sequenza delle operazioni,
l'indicazione del thread che ha effettuato l'operazione ed un
messaggio descrittivo. Le sigle dei thread che possono registrare una
azione nel log sono le seguenti:
\begin{itemize}
\item \verb|SSS|: \verb|MCSServer| thread (ovvero il thread principale
  che si occupa di ascoltare le richieste di connessione sul socket e
  generare nuovi thread non appena queste arrivano);
\item \verb|000|: \verb|MCSRoot| thread (ovvero il thread che provvede
  alle operazioni locali, indipendenti dall'attivit\`a dei client);
\item \verb|001| - \verb|100|: \verb|MCSClient| thread (ovvero i
  client che si occupano di ascoltare le richieste degli utenti e di
  soddisfarle). Il numero indica il Client identifier;
\end{itemize}

Ulteriori informazioni si possono trovare nel campo successivo. Questo
campo infatti, oltre a contenere un messaggio descrittivo dell'azione
pu\'o contenere sequenze di caratteri con significati particolari:
\begin{itemize}
\item \verb|< |: Command ricevuto dal client;
\item \verb|# |: Risposta inviata al client;
\item \verb|[at MCSClient.cc:573 (warn)]: #100Syntax error: aaaa| : \'E
  avvenuto un errore ignorabile nel file sorgente \verb|MCSClient.cc| alla
  riga 573;
\end{itemize}

Per quest'ultimo tipo di messaggio descrittivo gli errori possono
essere di tre tipi:
\begin{itemize}
\item \verb|(warn)|: \`e avvenuto un errore non grave;
\item \verb|(error)|: \`e avvenuto un errore grave;
\item \verb|(kill)|: \`e avvenuto un errore grave ed il thread che lo
  ha generato sar\`a distrutto;
\end{itemize}

\bigskip

Segue un esempio di log:
\footnotesize
\begin{verbatim}


My Customizable Server (MCS) ver. 0.0.2


YYYYMMDD HHMMSS ID-Log thr MESSAGE
-------- ------ ------ --- -----------------------------------------------------
20040729 163010 000001 MMM APPD set to: /home/gcalderone
20040729 163010 000002 MMM MCSServer                            created
20040729 163010 000003 MMM Creating new MCSRoot...
20040729 163010 000004 000 MCSClient                            created
20040729 163010 000005 MMM MCS Server running on port 6523
20040729 163017 000006 MMM Creating new MCSClient on ID=1...
20040729 163017 000007 001 MCSClient                            created
20040729 163017 000008 001 #001Welcome at glinux.local running calculator ver. 1.0
20040729 163020 000009 001 < cid
20040729 163020 000010 001 #015|1| (Client ID)
20040729 163022 000011 001 < aaa
20040729 163022 000012 001 [at MCSClient.cc:573 (warn)]: #100Syntax error: aaa
20040729 163024 000013 001 #099Bye.
20040729 163024 000014 001 MCSClient                            destroyed
Signal 2 received
20040729 163025 000015 MMM Killing root thread...
20040729 163025 000016 000 MCSClient                            destroyed
20040729 163025 000017 MMM Killing all clients...
20040729 163025 000018 MMM MCSServer                            destroyed
\end{verbatim}


\section{Linee guida per la scrittura di shell script}
I shell script sono normali programmi per la shell, che vengono
eseguiti in seguito ad una richiesta del client con il command
\verb|SCR|. Devono essere memorizzati nella directory
\verb|APPD/usr/scr| se sono destinati agli utenti, nella directory
\verb|APPD/adm/scr| se destinati all'amministratore di sistema.  Gli
shell script devono utilizzare i parametri della riga di command come
ingresso (non devono mai leggere da standard input altrimenti viene
bloccata l'esecuzione del thread), e lo standard output come
uscita. Inoltre devono sempre tornare un codice di uscita uguale a
zero, in caso di esecuzione corretta, o diverso da zero se \`e
avvenuto un errore, in tal caso \`e consigliabile scrivere un
messaggio indicativo su standard error. \`E utile tenere presente, per
evitare problemi di sicurezza, che gli script vengono eseguiti con i
diritti dell'utente \verb|SYS_USER|.

\section{Linee guida per la scrittura di SQL script e batch script}
I SQL script sono un elenco di istruzioni SQL che verranno eseguite
una dopo l'altra, in risposta al command \verb|SQL|. Tali file devono
essere memorizzati nella directory \verb|APPD/usr/sql| se sono
destinati agli utenti, nella directory \verb|APPD/adm/sql| se
destinati all'amministratore di sistema. Le istruzioni che compongono
gli script sono separate dal carattere punto e virgola, possono quindi
essere scritte su pi\`u righe. Se anche una sola di esse provoca un
errore l'esecuzione dello script verr\'a interrotta. I SQL script
dispongono anche di un meccanismo di sostituzione dei parametri, \'e
possibile infatti utilizzare le stringhe \verb|$1|, \verb|$2|, ecc...,
per riferirsi rispettivamente al primo, secondo parametro, ecc...
passati al command \verb|SQL| dopo l'indicazione del nome dello
script. Per quanto riguarda i batch script valgono esattamente le
stesse considerazioni, soltanto che in questo caso le singole
istruzioni sono comandi \mcs. Anche in questo caso si possono
utilizzare i parametri per la sostituzione. \`E consigliabile non
eseguire comandi \verb|BAT| all'interno di batch script.






\chapter{Examples}
In questo capitolo presentiamo alcuni semplici esempi allo scopo di
rendere pi\`u chiaro l'utilizzo delle classi fornite da
\mcs. Per un esempio pi\`u complesso si veda la documentazione
relativa a \verb|SDAMS|. Per collegarsi ai server mostrati nei
seguenti esempi basta utilizzare un semplice client telnet con il
seguente command:

\bigskip
\verb|telnet localhost <PORT>|
\bigskip\newline
%
Dove \verb|<PORT>| va sostituito con il numero di porta utilizzato.


\section{Un semplice echo server}
L'esempio pi\`u semplice da cui partire per dimostrare come funziona
un server \`e un cosiddetto echo server, ovvero un programma che
risponde con la string a che abbiamo inviato. Tale esempio viene
installato nella directory degli esempi. Si noti che abbiamo indicato
tutte quelle porzioni di codice che sono obbligatorie. Segue il codice
sorgente: \newline
%
\footnotesize
% Remember to use the lgrind style

\Head{}
\File{echosrv.cc}{2007}{3}{22}{0:33}{502}
\L{\LB{\K{\#include}_\<\V{mcs}.\V{hh}\>}}
\L{\LB{\K{using}_\K{namespace}_\V{mcs};}}
\L{\LB{}}
\L{\LB{\C{}//Customize_the_user_thread_class\CE{}}}
\L{\LB{\V{MCS\_CUSTOM\_USER}(\V{User});}}
\index{hk\_exec}\Proc{hk\_exec}\L{\LB{\V{RetValue}_\V{hk\_exec}(\V{CommandParser}*_\V{cmd},_\K{bool}\&_\V{cmd\_executed})}}
\L{\LB{\{}}
\L{\LB{}\Tab{2}{\V{string}_\V{s}_=_\S{}\3You_said:_\3\SE{}_+_\V{cmd}\-\!\>\V{cline}();}}
\L{\LB{}\Tab{2}{\K{return}_\V{Send}(_\V{MCS\_OK}(_\V{MSG\_OUT},_\V{s})_);}}
\L{\LB{\}}}
\L{\LB{\V{MCS\_CUSTOM\_USER\_END}(\V{User});}}
\L{\LB{}}
\L{\LB{\C{}//Declare_a_customized_server_named_\3echosrv\3,\CE{}}}
\L{\LB{\C{}//using_the_customized_user_class_\3User\3_(defined_above)\CE{}}}
\L{\LB{\V{MCS\_CUSTOM\_SERVER}(\V{User},_\V{LocalThread});}}
\L{\LB{}}
\index{main}\Proc{main}\L{\LB{\K{int}_\V{main}(\K{int}_\V{argc},_\K{char}_*\V{argv}[\,])_\{}}
\L{\LB{}\Tab{2}{\V{Env}*_\V{env}_=_\V{mcsCustomStart}(\S{}\3echosrv\3\SE{});}}
\L{\LB{}\Tab{2}{\V{mcsWait}(\V{env});}}
\L{\LB{\}}}

\normalsize
%
L'analisi di questo codice \`e abbastanza intuituiva per coloro i
quali hanno letto i capitoli precedenti. Per compilare questo
programma basta eseguire il programma \verb|make|, il cui
\verb|Makefile| si trova nella stessa directory dell'esempio. Per
eseguirlo basta dare il command con il nome dell'eseguibile. Si noti
che non \`e necessario creare alcuna struttura di directory o tabella
di database.

\section{Un calcolatore, comandi personalizzati}
In questo esempio vediamo come sia possibile creare comandi utente
personalizzati sfruttando l'hook \verb|hk_exec|. A tale scopo viene
utilizzato il metodo \verb|chk| della classe \verb|MCSHandleCmd|. Un
altro importante concetto mostrato qui \`e la gestione dei segnali,
attraverso le routine \verb|set_signal| e \verb|handle_signal|. Tale
gestione si rende necessaria per evitare che a causa di una
interruzione il server venga chiuso senza aver prima liberato tutte le
risorse utilizzate. Segue il codice sorgente: \newline
%
\footnotesize
\input{../share/examples/calculator.cc}
\normalsize
%

\section{Una agenda dei contatti: accesso al database}
In questo esempio mostriamo come sia semplice creare una agenda dei
contatti minimale. Vengono definiti due nuovi comandi: \verb|ins| e
\verb|search|. Con il primo si inizia la sequenza di inserimento di un
nuovo contatto e verranno richieste le singole informazioni. Con il
secondo, che prevede un parametro alfanumerico, verr\`a cercato un
contatto il cui cognome contiene il parametro dato. Nel caso in cui
uno o pi\`u contatti venissero trovati, verranno inviate tutte le
informazioni relative al client. Notiamo che in questo esempio, oltre
a derivare la classe \verb|MCSClient|, deriviamo anche la
\verb|MCSRoot| per far s\`i che la prima volta che il programma viene
eseguito vengano creati il database e la tabella necessari. Affinch\`e
il programma funzioni correttamente \`e necessario inserire il nome
utente e la password di un utente MySQL esistente, nel codice di
\verb|addrbook.cc|. Segue il codice sorgente: \newline
%
\footnotesize
\input{../share/examples/addrbook.cc}
\normalsize
%

\section{Esempio di utilizzo di programmi esterni}

\section{Punto di partenza per una nuova applicazione}
Per scrivere una nuova applicazione \`e sempre conveniente utilizzare
lo script \verb|mcs_newapp|. Per quanto riguarda il codice sorgente,
tale script copia il file \verb|mcstest.cc| presente in
\verb|sharedir/examples/mcstest|, che rappresenta un buon punto di
partenza per la scrittura di una nuova applicazione poich\`e esso
presenta gi\`a lo scheletro per la derivazione delle classi
\verb|MCSClient| e \verb|MCSRoot|, nonch\`e la definizione dei simboli
necessari, la gestione dei segnali di sistema, la gestione delle
opzioni della riga di command per l'attivazione dei singoli switch di
\verb|MCSEnv| e  la creazione degli oggetti necessari. Qualunque
applicazione dovrebbe seguire questo modello, eventualmente eliminando
ci\`o che non serve.
